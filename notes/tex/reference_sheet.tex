
%%%%%%%%%%%%%%%%%%%%%%%%%%%%%%%%%%%%%%%%%%%%%%%%%%%%%%%%%%%%%%%%%%%%%%
% Reference Sheet For Time Series (MSDS6373)
%
%
% Sources:
% https://github.com/sjmiller8182/LearningTS/blob/master/notes/tex/reference_sheet.tex
% https://github.com/sjmiller8182/LearningTS/blob/master/notes/reference_sheet.pdf
%
% Provided freely under the MIT license
%
% Feel free to distribute, but keep this and the following referral.
% 
%%%%%%%%%%%%%%%%%%%%%%%%%%%%%%%%%%%%%%%%%%%%%%%%%%%%%%%%%%%%%%%%%%%%%%
%%%%%%%%%%%%%%%%%%%%%%%%%%%%%%%%%%%%%%%%%%%%%%%%%%%%%%%%%%%%%%%%%%%%%%
% writeLaTeX Example: A quick guide to LaTeX
%
% Source: Dave Richeson (divisbyzero.com), Dickinson College
% 
% A one-size-fits-all LaTeX cheat sheet. Kept to two pages, so it 
% can be printed (double-sided) on one piece of paper
% 
% Feel free to distribute this example, but please keep the referral
% to divisbyzero.com
% 
%%%%%%%%%%%%%%%%%%%%%%%%%%%%%%%%%%%%%%%%%%%%%%%%%%%%%%%%%%%%%%%%%%%%%%

\documentclass[a4paper]{article}
\usepackage{amssymb,amsmath,amsthm,amsfonts}
\usepackage{multicol,multirow}
\usepackage{calc}
\usepackage{ifthen}
\usepackage[landscape]{geometry}
\usepackage[colorlinks=true,citecolor=blue,linkcolor=blue]{hyperref}


\ifthenelse{\lengthtest { \paperwidth = 11in}}
{ \geometry{top=.5in,left=.5in,right=.5in,bottom=.5in} }
{\ifthenelse{ \lengthtest{ \paperwidth = 297mm}}
	{\geometry{top=1cm,left=1cm,right=1cm,bottom=1cm} }
	{\geometry{top=1cm,left=1cm,right=1cm,bottom=1cm} }
}

\pagestyle{empty}
\makeatletter
\renewcommand{\section}{\@startsection{section}{1}{0mm}%
	{-1ex plus -.5ex minus -.2ex}%
	{0.5ex plus .2ex}%x
	{\normalfont\large\bfseries}}
\renewcommand{\subsection}{\@startsection{subsection}{2}{0mm}%
	{-1explus -.5ex minus -.2ex}%
	{0.5ex plus .2ex}%
	{\normalfont\normalsize\bfseries}}
\renewcommand{\subsubsection}{\@startsection{subsubsection}{3}{0mm}%
	{-1ex plus -.5ex minus -.2ex}%
	{1ex plus .2ex}%
	{\normalfont\small\bfseries}}
\makeatother
\setcounter{secnumdepth}{0}
\setlength{\parindent}{0pt}
\setlength{\parskip}{0pt plus 0.5ex}
% -----------------------------------------------------------------------

\title{Time Series Analysis}

\begin{document}
	
	\raggedright
	\footnotesize
	
	\begin{center}
		\Large{\textbf{Time Series Analysis}} \\
	\end{center}
	\begin{multicols}{3}
		\setlength{\premulticols}{1pt}
		\setlength{\postmulticols}{1pt}
		\setlength{\multicolsep}{1pt}
		\setlength{\columnsep}{2pt}
		
		\section{Basics}
		
		\begin{align*}
		\gamma_0 = \frac{1}{N} \sum_{t=1}^{N} (x_t - \bar{x})^2 \\
		\gamma_k = \frac{1}{N} \sum_{t=1}^{N-k} (x_t - \bar{x}) (x_{t+k} - \bar{x}) \\
		\rho_0 = 1 \\
		\rho_k = \frac{\gamma_k}{\gamma_0} \\
		\hat{\sigma}_X^2 = \frac{\hat{\sigma}^2}{n} \left( 1 + 2 
			\sum_{k=1}^{n-1} \left( 1 - \frac{|k|}{n} \hat{\rho_k} \right) \right) \\
		CI: \bar{X} \pm 1.96 \sqrt{\frac{\hat{\sigma}^2}{n} 
			\left( 1 + 2 
			\sum_{k=1}^{n-1} \left( 1 - \frac{|k|}{n} \hat{\rho_k} \right) \right)}
		\end{align*}
		
		\section{Autoregressive Models}
		
		\subsection{AR(1) Models}
			\begin{align*}
			X_t- \phi X_{t-1} = a_t \\
			(1 - \phi z) = 0 \\
			\rho_0 = 1 \\
			\rho_k = \phi_1^k \\
			\sigma_X^2 = \frac{\sigma_a^2}{1 - \phi_1^2}
			\end{align*}
		
		\subsection{AR(1) Properties}
		\begin{itemize}
			\item Positive $\phi$
			\begin{itemize}
				\item Realizations appear to be wandering (aperiodic)
				\item Autocorrelations are damped exponentials
				\item Spectral densities have peaks at zero
			\end{itemize}
			\item Negative $\phi$
			\begin{itemize}
				\item Realizations appear to be oscillating
				\item Autocorrelations are damped oscillating exponentials
				\item Spectral densities have peaks at $f = 0.5$
			\end{itemize}
		\end{itemize}
		
		\subsection{AR(2) Models}
			\begin{align*}
			X_t - \phi_1 X_{t-1} - \phi_2 X_{t-2} = a_t \\
			(1 - \phi_1 z - \phi_2 z^2) = 0 \\
			\rho_0 = 1 \\
			\rho_1 = \frac{\phi_1}{1 - \phi_2} \\
			\rho_2 = \frac{\phi_1^2 + \phi_2 - \phi_2^2}{1 - \phi_2} \\
			\sigma_X^2 = \frac{1}{1 - \phi_1\rho_1 - \phi_2\rho_2}
			\end{align*}
		
		\subsection{AR(2) Properties}
		\begin{itemize}
			\item Two Real Roots - Both Pos
			\begin{itemize}
				\item The realization will appear to be wandering
				\item The autocorrelations will be exponentially damped
				\item There will be a peak at 0
			\end{itemize}
			\item Two Real Roots - Both Neg
			\begin{itemize}
				\item The realization will appear to be oscillating
				\item The autocorrelations will be damped oscillating exponentials
				\item There will be a peak at 0.5
			\end{itemize}
			\item Two Real Roots - One Each
			\begin{itemize}
				\item The realization will appear to be wandering and an oscillation will run on the realization
				\item The autocorrelations will be exponentially damped with a hint of oscillation
				\item There will be peaks at 0 and 0.5 in the spectal density
			\end{itemize}
			\item One Complex
			\begin{itemize}
				\item The realization will appear to have a pseudo-cyclic behavior with a cycle length of $\frac{1}{f_0}$
				\item The autocorrelations will be damped exponentials oscillating in a sinusoid envelope with a frequency of $f_0$
				\item There will be a peak at $f_0$ (between 0 and 0.5)
			\end{itemize}
		\end{itemize}
		
		$$
		f_0 = \frac{1}{2\pi}cos^{-1} \left( \frac{\phi_1}{2 \sqrt{-\phi_2}}  \right)
		$$
		
		\subsection{AR(p) Models}
		
		\begin{align*}
		X_t - \beta + \phi_1 X_{t-1} + \phi_2 X_{t-2} + ... + \phi_2 X_{t-p} = a_t \\
		x_t - \phi_1 B X_{t} - \phi_2 B^2 X_{t} - ... - \phi_p B^p X_{t} = a_t
		\end{align*}
		
		\subsubsection{Key Concepts}
		\begin{itemize}
			\item An AR(p) model is stationary if and only if all the roots of the characteristic equation are outside the unit circle.
			\item Any AR(p) characteristic equation can be numerically factored into 1st and 2nd order elements.
			\item These factors are interpreted as contributing AR(1) and AR(2) behaviors to the total behavior of the AR(p) model.
		\end{itemize}
		
		\subsubsection{Factor Contributions}
		AR(p) models reflect a contribution of AR(1) and AR(2) contributions.
		Roots that are close to the unit circle will be the dominate behavior.
		\begin{itemize}
		\item First order factors $( 1 - \phi_1 B)$
			\begin{itemize}
				\item Associated with real roots
				\item Contribute AR(1)-type behavior to the AR(p) model
				\item Associated with a system frequency of 0 if $\phi_1$ is positive or 0.5 if $\phi_1$ is negative
			\end{itemize}
		\item Second order factors $( 1 - \phi_1 B - \phi_2 B^2 )$
			\begin{itemize}
				\item Associated with complex roots
				\item Contribute cyclic AR(2) behavior to the AR(p) model
				\item Associated with a system frequency of $f_0$
			\end{itemize}
		\end{itemize}
	
	\section{Moving Average Models}
	
		\subsection{MA(1) Models}
		\begin{align*}
		X_t = a_t - \theta a_{t-1}\\
		(1 - \theta_1 z) = 0 \\
		\rho_0 = 1 \\
		\rho_1 = \frac{-\theta_1}{1 + \theta_1^2} \\
		\rho_k = 0 |_{k>1} \\
		\sigma_X^2 = \sigma_a^2 \left( 1 + \theta_1^2 \right)
		\end{align*}
		
		\subsection{MA(2) Models}
		\begin{align*}
		X_t = a_t - \theta_1 a_{t-1} - \theta_2 a_{t-2} \\
		(1 - \theta_1 z - \theta_2 z^2) = 0 \\
		\rho_0 = 1 \\
		\rho_1 = \frac{-\theta_1 + \theta_1 \theta_2}{1 + \theta_1^2 + \theta_2^2} \\
		\rho_2 = \frac{-\theta_2}{1 + \theta_1^2 + \theta_2^2} \\
		\rho_k = 0 |_{k>2} \\
		\sigma_X^2 = \sigma_a^2 \left( 1 + \theta_1^2 + \theta_2^2 \right)
		\end{align*}
		
		\subsection{MA(q) Models}
		\begin{align*}
		X_t = a_t-\theta_1 a_{t-1}-...-\theta_2 a_{t-q} \\
		x_t = a_t - \theta_1 B a_{t} - ... - \theta_q B^q X_{t}
		\end{align*}
		
			\subsubsection{Key Concepts}
			\begin{itemize}
				\item MA models are a finite GLP
				\item MA models are always stationary
				\item MA models are invertable iff all the roots are outside of the unit circle.
			\end{itemize}
	
		\subsection{MA Inversion}
		\begin{itemize}
			\item Real Root: use $1 / \theta$
			\item Complex Roots: use $\theta_1 = r_1^{-1} + r_2^{-1}$ and 
			$\theta_2 = -r_1^{-1} r_2^{-1}$
		\end{itemize}
	
	\section{ARMA(p,q) Models}
	
		\begin{align*}
		X_t = \beta +\phi_1 X_{t-1}+ ... +\phi_2 X_{t-p} = a_t-\theta_1 a_{t-1}-...-\theta_2 a_{t-q} \\
		x_t - \phi_1 B X_{t} - ... - \phi_p B^p X_{t} = a_t - \theta_1 B a_{t} - ... - \theta_q B^q X_{t}
		\end{align*}
		
		\subsection{Key Concepts}
		\begin{itemize}
			\item Valid when the model is stationary and invertable
			\begin{itemize}
				\item Stationary: roots of $\phi (z)$ are outside the unit circle
				\item Invertable: roots of $\theta (z)$ are outside the unit circle
			\end{itemize}
			\item $\phi (z)$ and $\theta (z)$ have no common factors (check)
		\end{itemize}

	\section{ARIMA}
		\subsection{General Form}
			\begin{align*}
			\phi \left( B \right) \left( 1 - B \right)^d X_t = \theta \left( B \right) a_t
			\end{align*}
			
		\subsection{Properties}
			\begin{itemize}
				\item The roots on the unit circle dominate the behavior of the realization
				\item The autocorrelations are defined to have a magnitude of 1  ($\rho_k = 1$)
				\item The variance of ARIMA is not well defined
			\end{itemize}

	\section{ARUMA}
	
	ARUMA is an generalization of ARIMA that includes a term or term(s) for seasonality.
	
		\begin{align*}
		\phi \left( B \right) \left( 1 - B \right)^d \left( 1- B^s \right) X_t = \theta \left( B \right) a_t
		\end{align*}
		
		\subsection{Monthly Seasonality}
		\begin{align*}
		\left( 1 - B^4 \right) = (1-B) (1+ B) \left( 1 + B^2 \right)
		\end{align*}
		
	\section{General Linear Processes}
	
		\subsection{General Form}
		Use \texttt{psi.weights.wge} to calculate $\psi$s
			\begin{align*}
			X_t - \mu = \sum_{j=0}^\infty \psi_j a_{t-j}
			\end{align*}
		
		\begin{itemize}
			\item An MA model can be represented as finite GLP
			\item An AR model can be represented as infinite GLP
		\end{itemize}
	
	\section{Forecasting}
	
		\subsection{Notation}
			\begin{itemize}
				\item $t_0$ - origin of the forecast
				\item $l$ - number of time units to forecast (lead time)
				\item $\hat{X}_{t_0} \left( l \right)$ - the forecast of 
				$X_{t_0 + l}$ given data up to $t_0$
			\end{itemize}
	
		\subsection{ARMA Forecasting}
		Use \texttt{fore.arma.wge()} for forecasting.
			\begin{align*}
			\hat{X}_{t_0} \left( {l} \right) = 
			\sum_{i=1}^p \phi_i \hat{X}_{t_0} \left( l - i \right) - 
			\sum_{j=1}^q \theta_j \hat{a}_{t_0 + l - j} + 
			\bar{x} \left[ 1 - \sum_{i = 1}^p \phi_i \right]
			\end{align*}
			
			\begin{align*}
			\hat{\sigma}^2_a = \frac{1}{n-p} \sum_{t=p+1}^{n} \hat{a}^2_t
			\end{align*}
			
			\subsubsection{Facts}
				\begin{align*}
				e_{t_0} \left( l \right) = X_{t_0 + l} - \hat{X}_{t_0} \left( l \right) \\
				var \left[e_{t_0} \left( l \right)\right] = 
				\sigma^2_a \sum_{j=0}^{l-1} \psi_j^2 \\
				FI: \hat{X}_{t_0} \left( {l} \right) \pm z_{1-\alpha/2} \sigma_a 
				\left[ \sum_{k=0}^{l-1} \psi_k^2 \right]^{1/2}
				\end{align*}
	
		\subsection{ARIMA Forecasting}
		Use \texttt{fore.aruma.wge()} for forecasting.
			\begin{itemize}
				\item Limits become unbounded as l increases
				\item A factor of $(1-B)$ does not forecast a trend.
				An order of $d > 1$ is required to forecast a trend.
			\end{itemize}
	
		\subsection{ARIMA with Seasonality Forecasting}
			The forecast for step $l$ is same as the last $s$ value.
			Use \texttt{fore.aruma.wge()} for forecasting.
			\begin{itemize}
				\item Limits become unbounded as l increases
				\item A factor of $(1-B)$ does not forecast a trend.
				An order of $d > 1$ is required to forecast a trend.
				\item $ \left(1-B\right) \left(1-B^s\right) = a_t $ is called an airline model.
			\end{itemize}
		
		\subsection{Linear Forecasting}
			Use \texttt{fore.sigplusnoise.wge()} for forecasting.
			\begin{itemize}
				\item Fit an OLS to $X_t$
				\item Fit an AR(p) to the residuals ($Z_t$)
			\end{itemize}
		
			$$
			\hat{X}_{t_0} \left( {l} \right) =
			b_0 + b_1 t + \hat{Z}_{t_0} \left( {l} \right)
			$$
		
			$$
			FI: b_0 + b_1 t + \hat{Z}_{t_0} \left( {l} \right) \pm 
			z_{1-\alpha/2} \hat{\sigma}_a \left[ \sum_{k=0}^{l-1} \psi_k^2 \right]^{1/2}
			$$
		
	\section{Filtering}
	
		Filters transform time series.
		
		$$
		Z_t \rightarrow H(B) \rightarrow X_t
		$$
		
		$$
		X(t) =  Z(t) H(B)
		$$
		
		There are four basic types of filters.
		
		\begin{itemize}
			\item High pass - filters out low frequencies
			\item Low pass - filters out high frequencies
			\item Band pass - filters out frequencies outside the band
			\item Band stop - filters out frequencies inside the band
		\end{itemize}
	
		\subsection{Difference Filter}
			The first order difference is expressed by the following
			\begin{align*}
			X_t = Z_t - Z_{t-1} \\
			H(B) = B^0 - B
			\end{align*}
			This is a high pass filter.
			
		\subsection{Moving Average Filter}
			A 5-point moving average filter can be expressed as 
			\begin{align*}
			X_t = \frac{Z_{t+2} + Z_{t+1} + Z_t + Z_{t-1} + Z_{t-2}}{5} \\
			H(B) = \frac{B^{-2} + B^{-1} + B^0 + B + B^2}{5}
			\end{align*}
			This is a low pass filter.
			
		\subsection{Band-Type Filter}
			High pass and low pass filters can be combined to produce band pass and band stop filters.
	\end{multicols}
	
\end{document}