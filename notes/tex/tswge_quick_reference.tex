
%%%%%%%%%%%%%%%%%%%%%%%%%%%%%%%%%%%%%%%%%%%%%%%%%%%%%%%%%%%%%%%%%%%%%%
% Quick Reference for tswge
%
%
% Sources:
% https://github.com/sjmiller8182/LearningTS/blob/master/notes/tex/tswge_quick_reference.tex
% https://github.com/sjmiller8182/LearningTS/blob/master/notes/tswge_quick_reference.pdf
%
% Provided freely under the MIT license
%
% Feel free to distribute, but keep this and the following referral.
% 
%%%%%%%%%%%%%%%%%%%%%%%%%%%%%%%%%%%%%%%%%%%%%%%%%%%%%%%%%%%%%%%%%%%%%%
%%%%%%%%%%%%%%%%%%%%%%%%%%%%%%%%%%%%%%%%%%%%%%%%%%%%%%%%%%%%%%%%%%%%%%
% writeLaTeX Example: A quick guide to LaTeX
%
% Source: Dave Richeson (divisbyzero.com), Dickinson College
% 
% A one-size-fits-all LaTeX cheat sheet. Kept to two pages, so it 
% can be printed (double-sided) on one piece of paper
% 
% Feel free to distribute this example, but please keep the referral
% to divisbyzero.com
% 
%%%%%%%%%%%%%%%%%%%%%%%%%%%%%%%%%%%%%%%%%%%%%%%%%%%%%%%%%%%%%%%%%%%%%%

\documentclass[a4paper]{article}
\usepackage{amssymb,amsmath,amsthm,amsfonts}
\usepackage{multicol,multirow}
\usepackage{calc}
\usepackage{ifthen}
\usepackage[landscape]{geometry}
\usepackage[colorlinks=true,citecolor=blue,linkcolor=blue]{hyperref}


\ifthenelse{\lengthtest { \paperwidth = 11in}}
{ \geometry{top=.5in,left=.5in,right=.5in,bottom=.5in} }
{\ifthenelse{ \lengthtest{ \paperwidth = 297mm}}
	{\geometry{top=1cm,left=1cm,right=1cm,bottom=1cm} }
	{\geometry{top=1cm,left=1cm,right=1cm,bottom=1cm} }
}

\pagestyle{empty}
\makeatletter
\renewcommand{\section}{\@startsection{section}{1}{0mm}%
	{-1ex plus -.5ex minus -.2ex}%
	{0.5ex plus .2ex}%x
	{\normalfont\large\bfseries}}
\renewcommand{\subsection}{\@startsection{subsection}{2}{0mm}%
	{-1explus -.5ex minus -.2ex}%
	{0.5ex plus .2ex}%
	{\normalfont\normalsize\bfseries}}
\renewcommand{\subsubsection}{\@startsection{subsubsection}{3}{0mm}%
	{-1ex plus -.5ex minus -.2ex}%
	{1ex plus .2ex}%
	{\normalfont\small\bfseries}}
\makeatother
\setcounter{secnumdepth}{0}
\setlength{\parindent}{0pt}
\setlength{\parskip}{0pt plus 0.5ex}
% -----------------------------------------------------------------------

\title{tswge Quick Reference}

\begin{document}
	
	\raggedright
	\footnotesize
	
	\begin{center}
		\Large{\textbf{\texttt{tswge} Quick Reference}} \\
	\end{center}
	\begin{multicols}{3}
		\setlength{\premulticols}{1pt}
		\setlength{\postmulticols}{1pt}
		\setlength{\multicolsep}{1pt}
		\setlength{\columnsep}{2pt}
		
		\section{Plotting}
		
		\begin{itemize}
			\item \texttt{plotts.wge} - plot a realization
			\item \texttt{plotts.sample.wge} - 4 sample plots (realization, parzen window, periodogram, spectral density)
			\item \texttt{plotts.parzen.wge} - parzen window (can be truncated)
			\item \texttt{plotts.true.wge} - true ACF and true spectral density
		\end{itemize}

		\section{Factor Analysis}
		
		\begin{itemize}
			\item \texttt{factor.wge} - factor characteristic equations into 1st and 2nd order components
			\item \texttt{factor.comp.wge} - decompose a sampled signal into factors
			\item \texttt{mult.wge} - multiply together factors (up to 6)
			\begin{itemize}
				\item Use \texttt{\$model.coef} to access the coefficients
			\end{itemize}
		\end{itemize}
		
		\section{Simulation}
		
		\begin{itemize}
			\item \texttt{gen.sigplusnoise.wge} - linear or sinusiod signal with normal noise
			\begin{itemize}
				\item Linear model: \texttt{b0, b1}
				\item Sinusiod: \texttt{coef, freq, psi} (2 component vectors)
				\item AR noise: \texttt{phi}
			\end{itemize}
			\item \texttt{gen.arma.wge} - arma signals
				\begin{itemize}
					\item \texttt{p}: AR vector
					\item \texttt{q} MA vector
				\end{itemize}
			\item \texttt{gen.arima.wge} - arima signals
				\begin{itemize}
					\item \texttt{p}: AR vector
					\item \texttt{q}: MA vector
					\item \texttt{d}: Difference order
				\end{itemize}
			\item \texttt{gen.aruma.wge} - aruma signals
				\begin{itemize}
					\item \texttt{p}: AR vector
					\item \texttt{q}: MA vector
					\item \texttt{d}: Difference order
					\item \texttt{s}: Seasonality
					\item \texttt{lambda}: Other non-stationary components
				\end{itemize}
		\end{itemize}
		
		\section{Forecast}
		
		\begin{itemize}
			\item \texttt{fore.arma.wge} - forecast arma signals
			\begin{itemize}
				\item \texttt{x}: The series
				\item \texttt{theta}: MA vector
				\item \texttt{phi}: AR vector
				\item \texttt{n.ahead}: Number of forecast steps
				\item \texttt{lastn}: Use last \texttt{n.ahead} values to validate forecast
			\end{itemize}
			\item \texttt{fore.aruma.wge} - forecast aruma signals
			\begin{itemize}
				\item \texttt{x}: The series
				\item \texttt{theta}: MA vector
				\item \texttt{phi}: AR vector
				\item \texttt{d}: Number of difference factors
				\item \texttt{s}: Seasonality factor
				\item \texttt{n.ahead}: Number of forecast steps
				\item \texttt{lastn}: Use last \texttt{n.ahead} values to validate forecast
			\end{itemize}
		\end{itemize}

		\section{Model Fitting}
		
		\begin{itemize}
			\item \texttt{aic5.wge} - fit models for given orders and return top 5
				\begin{itemize}
					\item \texttt{x}: The series
					\item \texttt{p}: Vector of orders
					\item \texttt{q}: Vector of orders
					\item \texttt{type}: Penalty \{ \texttt{'AIC'} (default), \texttt{'AICC'}, \texttt{'BIC'} \}
				\end{itemize}
			\item \texttt{est.ar.wge} - fit an AR model
				\begin{itemize}
					\item \texttt{x}: The series
					\item \texttt{p}: AR model to fit
					\item \texttt{type}: Fit method \{ \texttt{'mle'} (default), \texttt{'burg'}, \texttt{'yw'} \}
				\end{itemize}
			\item \texttt{est.arma.wge} - fit an AR model
				\begin{itemize}
					\item \texttt{x}: The series
					\item \texttt{p}: AR order to fit
					\item \texttt{q}: MA order to fit
				\end{itemize}
		\end{itemize}
		
		\section{Filtering}
		
		\begin{itemize}
			\item \texttt{artrans.wge} - apply an AR-type transformation to a series
			\begin{itemize}
				\item First order difference: \texttt{artrans.wge(x, phi.tr = 1)}
				\item Subtract monthly difference: \texttt{artrans.wge(x, phi.tr = c(rep(0,11),1))}
			\end{itemize}
			\item \texttt{butterworth.wge} - apply a butterworth filter a time series
			\item \texttt{stats::filter} - general function for filtering e.g.
			5-pt MA Filter: \texttt{stats::filter(X\_t, rep(1,5)/5)}
		\end{itemize}
		
	\end{multicols}
	
\end{document}