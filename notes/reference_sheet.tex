
%%%%%%%%%%%%%%%%%%%%%%%%%%%%%%%%%%%%%%%%%%%%%%%%%%%%%%%%%%%%%%%%%%%%%%
% writeLaTeX Example: A quick guide to LaTeX
%
% Source: Dave Richeson (divisbyzero.com), Dickinson College
% 
% A one-size-fits-all LaTeX cheat sheet. Kept to two pages, so it 
% can be printed (double-sided) on one piece of paper
% 
% Feel free to distribute this example, but please keep the referral
% to divisbyzero.com
% 
%%%%%%%%%%%%%%%%%%%%%%%%%%%%%%%%%%%%%%%%%%%%%%%%%%%%%%%%%%%%%%%%%%%%%%
% How to use writeLaTeX: 
%
% You edit the source code here on the left, and the preview on the
% right shows you the result within a few seconds.
%
% Bookmark this page and share the URL with your co-authors. They can
% edit at the same time!
%
% You can upload figures, bibliographies, custom classes and
% styles using the files menu.
%
% If you're new to LaTeX, the wikibook is a great place to start:
% http://en.wikibooks.org/wiki/LaTeX
%
%%%%%%%%%%%%%%%%%%%%%%%%%%%%%%%%%%%%%%%%%%%%%%%%%%%%%%%%%%%%%%%%%%%%%%

\documentclass[a4paper]{article}
\usepackage{amssymb,amsmath,amsthm,amsfonts}
\usepackage{multicol,multirow}
\usepackage{calc}
\usepackage{ifthen}
\usepackage[landscape]{geometry}
\usepackage[colorlinks=true,citecolor=blue,linkcolor=blue]{hyperref}


\ifthenelse{\lengthtest { \paperwidth = 11in}}
{ \geometry{top=.5in,left=.5in,right=.5in,bottom=.5in} }
{\ifthenelse{ \lengthtest{ \paperwidth = 297mm}}
	{\geometry{top=1cm,left=1cm,right=1cm,bottom=1cm} }
	{\geometry{top=1cm,left=1cm,right=1cm,bottom=1cm} }
}

\pagestyle{empty}
\makeatletter
\renewcommand{\section}{\@startsection{section}{1}{0mm}%
	{-1ex plus -.5ex minus -.2ex}%
	{0.5ex plus .2ex}%x
	{\normalfont\large\bfseries}}
\renewcommand{\subsection}{\@startsection{subsection}{2}{0mm}%
	{-1explus -.5ex minus -.2ex}%
	{0.5ex plus .2ex}%
	{\normalfont\normalsize\bfseries}}
\renewcommand{\subsubsection}{\@startsection{subsubsection}{3}{0mm}%
	{-1ex plus -.5ex minus -.2ex}%
	{1ex plus .2ex}%
	{\normalfont\small\bfseries}}
\makeatother
\setcounter{secnumdepth}{0}
\setlength{\parindent}{0pt}
\setlength{\parskip}{0pt plus 0.5ex}
% -----------------------------------------------------------------------

\title{Griffith QM Time Dependent Perturbation Theory CheatSheet}

\begin{document}
	
	\raggedright
	\footnotesize
	
	\begin{center}
		\Large{\textbf{Time Series Analysis}} \\
	\end{center}
	\begin{multicols}{3}
		\setlength{\premulticols}{1pt}
		\setlength{\postmulticols}{1pt}
		\setlength{\multicolsep}{1pt}
		\setlength{\columnsep}{2pt}
		
		\section{Basics}
		
		\subsection{Sample Autocorrelations}
		\begin{align*}
		\gamma_0 = \frac{1}{N} \sum_{t=1}^{N} (x_t - \bar{x}) \\
		\gamma_k = \frac{1}{N} \sum_{t=1}^{N-k} (x_t - \bar{x}) (x_{t+k} - \bar{x}) \\
		\rho_0 = 1 \\
		\rho_k = \frac{\gamma_k}{\gamma_0}
		\end{align*}
		
		\section{Autoregressive Models}
		
		\subsection{AR(1)}
		\begin{align*}
		X_t- \phi X_{t-1} = a_t \\
		(1 - \phi z) = 0
		\end{align*}
		
		\subsection{AR(1) Properties}
		\begin{itemize}
			\item Positive $\phi$
			\begin{itemize}
				\item Realizations appear to be wandering (aperiodic)
				\item Autocorrelations are damped exponentials.
				\item Spectral densities have peaks at zero
			\end{itemize}
			\item Negative $\phi$
			\begin{itemize}
				\item Realizations appear to be oscillating
				\item Autocorrelations are damped oscillating exponentials
				\item Spectral densities have peaks at $f = 0.5$
			\end{itemize}
		\end{itemize}
		
		\subsection{AR(2) Models}
		\begin{align*}
		X_t - \phi_1 X_{t-1} - \phi_2 X_{t-2} = a_t \\
		(1 - \phi_1 z - \phi_2 z^2) = 0
		\end{align*}
		
		\subsection{AR(2) Properties}
		\begin{itemize}
			\item Two Real Roots - Both Pos
			\begin{itemize}
				\item The realization will appear to be wandering
				\item The autocorrelations will be exponentially damped
				\item There will be a peak at 0
			\end{itemize}
			\item Two Real Roots - Both Neg
			\begin{itemize}
				\item The realization will appear to be oscillating
				\item The autocorrelations will be damped oscillating exponentials
				\item There will be a peak at 0.5
			\end{itemize}
			\item Two Real Roots - One Each
			\begin{itemize}
				\item The realization will appear to be wandering and an oscillation will run on the realization
				\item The autocorrelations will be exponentially damped with a hint of oscillation
				\item There will be peaks at 0 and 0.5 in the spectal density
			\end{itemize}
			\item One Complex
			\begin{itemize}
				\item The realization will appear to have a pseudo-cyclic behavior with a cycle length of $\frac{1}{f_0}$
				\item The autocorrelations will be damped exponentials oscillating in a sinusoid envelope with a frequency of $f_0$
				\item There will be a peak at $f_0$ (between 0 and 0.5)
			\end{itemize}
		\end{itemize}
		
		$$
		f_0 = \frac{1}{2\pi}cos^{-1} \left( \frac{\phi_1}{2 \sqrt{-\phi_2}}  \right)
		$$
		
		\subsection{AR(p)}
		
		\begin{align*}
		X_t = \beta + \phi_1 X_{t-1} + \phi_2 X_{t-2} + ... + \phi_2 X_{t-p} + a_t \\
		x_t - \phi_1 B X_{t} - \phi_2 B^2 X_{t} - \phi_p B^p X_{t} = a_t
		\end{align*}
		
		\subsubsection{Key Concepts}
		\begin{itemize}
			\item An AR(p) model is stationary if and only if all the roots of the characteristic equation are outside the unit circle.
			\item Any AR(p) characteristic equation can be numerically factored into 1st and 2nd order elements.
			\item These factors are interpreted as contributing AR(1) and AR(2) behaviors to the total behavior of the AR(p) model.
		\end{itemize}
		
		\subsubsection{Factor Contributions}
		AR(p) models reflect a contribution of AR(1) and AR(2) contributions.
		Roots that are close to the unit circle will be the dominate behavior.
		\begin{itemize}
		\item First order factors $( 1 - \phi_1 B)$
			\begin{itemize}
				\item Associated with real roots
				\item Contribute AR(1)-type behavior to the AR(p) model
				\item Associated with a system frequency of 0 if $\phi_1$ is positive or 0.5 if $\phi_1$ is negative
			\end{itemize}
		\item Second order factors $( 1 - \phi_1 B - \phi_2 B^2 )$
			\begin{itemize}
				\item Associated with complex roots
				\item Contribute cyclic AR(2) behavior to the AR(p) model
				\item Associated with a system frequency of $f_0$
		\end{itemize}
		\end{itemize}
	\end{multicols}
	
\end{document}